
\documentclass[12pt]{article}
\usepackage[T1]{fontenc}
\usepackage{geometry}
\geometry{verbose,a4paper,tmargin=15mm,bmargin=30mm,lmargin=30mm,rmargin=20mm}
%\usepackage{graphics}
\usepackage{amsmath}
\usepackage{amsfonts}
\usepackage{amssymb}
\usepackage{amsthm}
\usepackage{graphicx}
\usepackage{subcaption}
\usepackage{amsmath}
\usepackage{cite}
\usepackage{lipsum}
\usepackage{nomencl}
\makenomenclature
\usepackage{bibentry}
\usepackage{subfig}
\usepackage{graphicx}
\usepackage{caption}
\usepackage{gensymb}
\usepackage{refstyle}
\usepackage{setspace}
\theoremstyle{definition}
\newtheorem{defn}{Definition}[section]
\setcounter{secnumdepth}{5}
\setcounter{tocdepth}{5}
\graphicspath{ {images/} }
\onehalfspacing

\makeatletter

\linespread{2}
%%%%%%%%%%%%%%%%%%%%%%%%%%%%%% LyX specific LaTeX commands.
\newcommand{\LyX}{L\kern-.1667em\lower.25em\hbox{Y}\kern-.125emX\spacefactor1000}

\makeatother

\begin{document}
\begin{titlepage}
\thispagestyle{empty}
\vspace*{0.7cm}
{\centering     
\large
{\Large\bf Compressed Sensing in Internet of Things}\\
\vspace{3cm}
\bf{Seminar Report}\\
\vspace{0.25cm}
\vspace{0.1cm}
\it
by \\
\vspace{.5cm}
\rm
{\large \bf {Nagma Samreen Khan}}\\
{\large \bf {153079030}}

\vspace{1cm}

{\it{under the guidance of}} \\
\vspace{.5cm}

\hspace{.05cm} {\large \bf {Prof. Kumar Appaiah}}\\
%\hspace{.05cm}
\vspace {0.5cm}
%\iitbseal\ \\

\begin{figure}[h] 
%\hspace{6cm}
%\vspace{5cm}
{\centering {\includegraphics[width=0.32\textwidth]{images/IITB_logo}}\par}
\end{figure} 

Department of Electrical Engineering \\ 
Indian Institute of Technology, Bombay\\ 
{\centering
\hspace{6.5cm}April 2016} 
}
\pagebreak 
\end{titlepage}

\begin{abstract}
The abstract goes here.
\end{abstract}
\section{Introduction}
  \subsection{Internet of Things}
  Internet of Things or commonly referred to as IoT defines a paradigm in which 'everyday' devices can be connected 
  together in a network and thus can communicate with each other over the Internet. Moreover the connected devices are equipped 
  with 'identifying', 'sensing' and 'processing' \cite{Whitmore-survey}
  capabilites which enable them to 
  acheive some objective or complete some assigned task. Nowadays IoT is widely used in applications like
  healthcare, utilities, transport, etc. The revolution in the field of IoT has enabled
  these devices to have interaction capabilites too i.e. they can be controlled, actuated and commanded.
  \cite{Gubbi-vision}. 
  \par The components of IoT broadly are Hardware, Software and Architecture. Hardware refers to the component/s
  or the device/s which constitue the system which has been designed to acheive some objective, Software refers roughly
  to the support which enables the devices in the system to exchange and process data inspite of their heterogeniety 
   %\cite{Whitmore-survey}
   and Architecture provides the specifications so that information transfer and processing
   and networking of devices can be done in a standard fashion  \cite{Whitmore-survey} \cite{Gubbi-vision}.
  \par In healthcare applications, IoT can be used to monitor a patient's health condtion over the Internet, and 
  in case of critical patients the sensors placed on the patient can transmit and make information available continuously
  to the doctors and patient's relatives which will lead to better monitoring and enable doctors to take timely action
  in case of emergencies. IoT has already made its mark in making 'smart homes' where sensors and actuators placed in 
  the house or building complex continuously monitor and control the resource consumption and also fulfil security needs.
  On a broader scale, IoT can be deployed to make 'smart cities' in which everything starting from traffic to the levels 
  of pollution can be monitored and controlled \cite{Whitmore-survey}. 
  \par Even in the industries IoT can be used for quality management, improving efficiency and tracking of goods. 
  In can be used in hazardous industries, say the mining industry, to improve the safety level
  by monitoring the condition inside the mine via sensors and continuously process the data to provide better disaster
  management and in-time warning systems. In transportation and logistics, the monitoring of movement of goods, say 
  using Radio Frequency Identification( RFID ) can be 
  made 'real-time' using IoT and thus providing better tracking \cite{Gubbi-vision}.
  \par These are just some examples of how IoT has affeced, is affecting and will affect our lives but one thing is clear
  that in the coming years IoT will lead to further betterment  in our lives through saving our time, money and resources.
  \subsection{Compressed Sensing}
  With increase in the number of connected devices around us, the amount of data transmitted over the Internet will naturally
  increase. Also with increase in the number of sensors, large amounts of data will be generated which will have to be 
  processed, stored and transmitted over the Internet in a timely fashion. So consumption of resources by the entire system
  will increase. This is where Compressed or Compressive Sensing comes in. 
  \par The simplest way to define Compressed Sensing is that we transmit much less information compared to what has been
  measured by our sensors and the actual measured information can be reconstructed with a small error at the receiver's end.
  This opens up a world of possibilities as it leads to saving time and resources and thus data transmission can be done
  much faster. The only prerequisite is that the data has to \textit{sparse} or \textit{compressible}
  in some transform domain. These concepts
  are defined mathematically in the next section, but here an example is presented to give a flavour of Compressed Sensing.
  Say we want to measure temperature at every point in a large room and for that purpose say we deploy 100 sensors distributed
  uniformly across the room. Now we know that temperature is a slowly varying quantity and thus will have mainly low
  frequency components. So if we take the Fourier Transform of the data then we can ignore the higher frequency coefficients
  and thus the measurements are compressible in Fourier Transform domain.
 
\section{Mathematical Background to Compressed Sensing}
\begin{defn}
 A vector $\theta$ $\in$ $\mathbb{R}^n$ is said to be \textit{k-sparse} if there are \textit{k} non-zero coefficients in $\theta$.
\end{defn}
\begin{defn}
 A vector $\theta$ $\in$ $\mathbb{R}^n$ is said to be \textit{compressible} if there are \textit{k} non-zero coefficients in $\theta$.
\end{defn}
Suppose \textit{x}$\in$ $\mathbb{R}^n$ is original data measured by the sensors which is known to be \textit{sparse} or 
\textit{compressible}
\section{Recovery Algorithms}
\section{Verification of Recovery Algorithms through Simulation}
\section{Conclusion}
The conclusion goes here.

The authors would like to thank...\cite{Whitmore-survey} \cite{Huang-DGS}



\bibliographystyle{plain}
\bibliography{seminar_report}


\end{document}


